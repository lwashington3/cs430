%! Author = Len Washington III
%! Date = 8/25/2023

% Preamble
\documentclass[12pt]{report}

\usepackage[3]{cs430recitation}
\usepackage{algpseudocode}

% Document
\begin{document}

%<*Recitation-3>
\subsection{After Lecture 05 \& 06} -- Answer any questions on HW1\\
Practice Problems (all taken from previous exams)
\begin{enumerate}[label=\arabic*.]
	\item What are the max number of levels in the recursion tree for this recurrence relation? \[ T(n) = T\left( \frac{n}{4} \right) + T\left( \frac{3n}{4} \right) + n \]
	\begin{enumerate}[label=\alph*)]
	    \item $\log_{4}(n)$
	    \item $\log_{2}(n)$
	    \item \oldanswer{$\log_{\frac{4}{3}}(n)$}
	    \item $\log_{\frac{1}{3}}(n)$
	\end{enumerate}
	\item Under what case of Master's Theorem will the recurrence relation of binary search fail?
	\begin{enumerate}[label=\choicelabel]
	    \item 1
		\item \oldanswer{2}
		\item 3
		\item It cannot be solved using Master's Theorem.
	\end{enumerate}
	\item What is the purpose of using randomized quick sort over standard quick sort?
	\begin{enumerate}[label=\choicelabel]
	    \item Improve the worst-case runtime
		\item \oldanswer{To eliminate the possibility that a particular input order will always yied worst-case runtime}
		\item To improve accuracy of output
		\item To improve average case time complexity
	\end{enumerate}
	\item The non-recursive work in quicksort is done in which step of the divide-conquer-combine algorithm?
	\begin{enumerate}[label=\choicelabel]
	    \item divide
		\item conquer
		\item combine
		\item none
	\end{enumerate}
	\item Give big-O bounds of T(n) in each of the following recurrences. Use induction, iteration or Master Theorem.
	\begin{enumerate}[label=\arabic{enumi}\choicelabel]
	    \item \[ T(n) = T(n-1) + n\mbox{;  }T(1)=O(1) \]\oldanswer{\begin{equation*}
	    \begin{aligned}
	    	T(n) &= n + T(n-1)\\
	    	T(n) &= n + n-1 + T(n-2)\\
	    	T(n) &= n + n-1 + n-2 + T(n-3)\\
	    	T(n) &= n + n-1 + n-2 + \dots + T(1)\\
	    	T(n) &= n + n-1 + n-2 + \dots + O(1)\\
	    	T(n) &= \sum_{1}^{n}n \\
	    	T(n) &= O\left( \frac{n^{2}+n}{2} \right) \\
	    	T(n) &= O(n^{2}) \\
	    \end{aligned}
	    \end{equation*}}
		\item \[ T(n) = 2T\left( \frac{n}{4} \right) + n^{\frac{1}{2}}\mbox{;  } T(1)=O(1) \]
		\item \[ T(n) = T\left( \frac{n}{4} \right) + T\left( \frac{n}{2} \right) + n^{2}\mbox{;  }T(1)=O(1)\]
	\end{enumerate}
	\item Throughout this course, we assume that parameter passing during procedure calls takes constant time, even if an N-element array is being passed. This assumption is valid in most systems because a pointer to the array is passed, not the array itself. This problem examines the implications of three parameter-passing strategies:
	\begin{enumerate}[label=\arabic*.]
	    \item An array is passed by pointer. Time = $\theta(1)$.
		\item An array is passed by copying. Time = $\theta(N)$, where $N$ is the size of the array.
		\item An array is passed by copying only the subrange that might be accessed by the called procedure. Time = $\theta(q - p + 1)$ if the subarray $A[p \dots q]$ is passed. Use $n= q - p + 1$, where $n$ is the size of the subarray passed.
	\end{enumerate}
	Consider the recursive binary search algorithm for finding a number in a sorted array. Give recurrences for the worst-case running times of binary search when arrays are passed using each of the three methods above, and give good upper bounds on the solutions of the recurrences. Let $N$ be the size of the original problem and n be the size of a subproblem. \oldanswer{ Binary search works by comparing the element for which you are searching to the elemenet at index $\frac{p-r}{2}$ of a subarray of size $n$, where $p$ is the first index of the subarray and $r$ is the last index (integer division is used). Therefore, the array passed inot binary search is continually divided in hald.}
	\oldanswer{\begin{enumerate}[label=\arabic*)]
	    \item \[ T(n) = T\left( \frac{n}{2} \right) + O(1)\mbox{ with } T(1) = O(1) \] The array is passed by pointer, which is constant time. Therefore, the time involved in tha
		\item
	\end{enumerate}}
	\item Use the definition of $\Theta$ and induction to prove that the recurrence $T(n) = T(N-1) + \theta(n)$ (worst case Quicksort) has the solution $T(n) = \Theta(n^{2})$. \oldanswer{Since we are not given any boundary conditions, we cann assume the basis step for the inductive proof. Assume the claim is true for $n=k$.\\$T(k) = \theta(k^{2})$, in other words assume $c_{1}k^{2} \leq T(k) \leq c_{2}k^{2}$ for some $c_{1} > 0$, $c_{2} > 0$ and $k$ large enough.\\Use that to prove the claim}
	\item What is you are sorting a collection of data that can have multiple entries of some of the values. When calling Quicksort's \Call{Partition}{$A$, $p$, $r$}, where do elements equal to the pivot end up and why?\\
		How could we modify Quicksort and Partition (write pseudocode) so that if we happen to partition on a pivot that had many duplicate values, we can improve the runtime of Quicksort by having smaller recursive calls?\oldanswer{\begin{algorithm}[H]
					\caption{Quicksort1}\label{alg:quicksort1}
					\begin{algorithmic}[1]
					\Function{Quicksort1}{$A,p,r$}\Comment{}
						\If{$p < r$}
							\State $(q_{1}, q_{2})$ = \Call{Parition1}{$A,p,r$}\Comment{two return values}
							\State \Call($A, p, q_{1}-1$)
						\EndIf
					\EndFunction
					\end{algorithmic}
				\end{algorithm} \begin{algorithm}[H]
						\caption{Partition1}\label{alg:partition1}
						\begin{algorithmic}[1]
						\Function{Partition1}{$A,p,r$}\Comment{Two return values}
							\State endLow $\gets p-1$, endEqual $\gets p-1$
							\State pivot $\gets$ A[r]
							\For{j=p to r-1}
								\State get function
							\EndFor
						\EndFunction
						\end{algorithmic}
					\end{algorithm}}
\end{enumerate}
%</Recitation-3>

\end{document}