%! Author = Len Washington III
%! Date = 8/25/2023

% Preamble
\documentclass[12pt]{report}

\usepackage[2]{cs430recitation}
\usepackage{algpseudocode}

% Document
\begin{document}

%<*Recitation-2>
\subsection{After Lecture 03 \& 04} -- Answer any questions on HW1\\
Practice Problems (all taken from previous exams)
\begin{enumerate}[label=\arabic*.]
    \item Which of the following is time complexity of fun()? \lstinputlisting[language=C,label={lst:fun}]{2_fun.c}
	\begin{enumerate}[label=\choicelabel]
	    \item $O(n)$
	    \item \answer{$O(n^{2})$}
	    \item $O(n\log n)$
	    \item $O(n\log(n)\log(n))$
	\end{enumerate}
	\item Consider the following function. What is the returned value of the above function? \lstinputlisting[language=C,label={lst:unknown}]{2_unknown.c}
	\begin{enumerate}[label=\choicelabel]
	    \item $\Theta(n^{2})$
	    \item \answer{$\mathbf{\Theta(n^{2}\log n)}$ The outer loop runs $n/2$ or $\Theta(n)$ times. The inner loop runs $O(\log n)$ times (Note that $j$ is multiplied by $2$ in every iteration). So the statement "$k = k + n/2$;" runs $\Theta(n\log n)$ times. The statement increases value of $k$ by $n/2$. So the value of k becomes $n/2\times\Theta(n\log n)$ which is $\Theta(n^2\log n)$.}
	    \item $\Theta(n^{3})$
	    \item $\Theta(n^{3}\log n)$
	\end{enumerate}
	\item What is the worst-case auxiliary space complexity (including stack space for recursion) of merge sort?
	\begin{enumerate}[label=\choicelabel]
	    \item $O(1)$
	    \item $O(\log n)$
	    \item \answer{$\mathbf{O(n)}$ The worst case is every item is split into its own array at the lowest level of divide.}
	    \item $O(n\log n)$
	\end{enumerate}
	\item Choose the incorrect statement about merge sort from the following:
	\begin{enumerate}[label=\choicelabel]
	    \item It is a comparison-based sort.
		\item \answer{It's runtime is dependent on input order.}
		\item It is not an in-place algorithm (all the operations are on the original array).
		\item It is a stable algorithm.
	\end{enumerate}
	\item Use the definition of big-O to prove or disprove.
	\begin{enumerate}[label=\arabic{enumi}\choicelabel]
	    \item is $2^{n+1}?=?O\left( 2^{n} \right)$\answer{True: \begin{equation*}
	    \begin{aligned}
	    	2^{n+1}&=C2^{n}\\
	    	2&=C\mbox{ if c }\geq 2\\
	    \end{aligned}
	    \end{equation*}}
	    \item is $2^{2n}?=?O\left( 2^{n} \right)$\answer{False: \begin{equation*}
	    \begin{aligned}
	    	2^{2n}&=C2^{n}\\
	    	2^{n}&=C\\
			\mbox{However } 2^{n} \mbox{ has an infinite range so it cannot be upperbounded}
	    \end{aligned}
	    \end{equation*}}
	\end{enumerate}
	\item Although merge sort runs in $\Theta(n\lg n)$ worst-case time and insertion sort runs in $\Theta(n^{2})$ worst-case time, the constant factors in insertion sort make it faster for small $n$. Thus, it makes sense to use insertion sort within merge sort when sub-problems become sufficiently small. Consider a modification to merge sort in which $n/k$ sub-lists of length $k$ are sorted using insertion sort and then merged using the standard merging mechanism, where $k$ is a value to be determined.
	\begin{enumerate}[label=\arabic{enumi}\choicelabel]
	    \item Show the $n/k$ sub-lists; each of length $k$, can be sorted by insertion sort in $\Theta(nk)$ worst-case time.\answer{\begin{equation*}
	    \begin{aligned}
	    	k &\rightarrow \Theta\left( k^{2} \right)\\
			\frac{n}{k} &\rightarrow \Theta\left( \frac{n}{k} \times k^{2} \right)=\Theta(nk)\\
	    \end{aligned}
	    \end{equation*}}
		\item Show that the sub-lists can be merged in $\Theta(n\lg \left( \frac{n}{k} \right))$ worst-case time.\answer{\begin{equation*}
		\begin{aligned}
			\frac{n}{k} \rightarrow \Theta\left( \frac{n}{2k} \right)\mbox{ for merging}\\
			\Theta(2k) \mbox{ m?? for } 2k \mbox{ elements}\\
			\Theta\left( \frac{n}{2k}\times2k \right)=\Theta\left( n \right)
			\Theta\left( \frac{n}{4k}\times4k \right)=\Theta\left( n \right)
		\end{aligned}
		\end{equation*}}
		\item Given that the modified algorithm runs in $\Theta(nk+n\lg\left( \frac{n}{k} \right))$ worst-case time, what is the largest asymptotic ($\Theta$-notation) value of $k$ as a function of $n$ for which the modified algorithm has the same asymptotic running time as standard merge sort?\answer{\begin{equation*}
		\begin{aligned}
			nk + n\lg\left( \frac{n}{k} \right) &\leq n\lg n\\
			k + \lg\left( \frac{n}{k} \right) &\leq \lg n\\
			k + \lg(n) - \lg(k) &\leq \lg n\\
			k - \lg(k) &\leq 0\\
			k &\leq \lg(k)\\
		\end{aligned}
		\end{equation*}}
		\item How should $k$ be chosen in practice?
	\end{enumerate}
	\item The Fibonacci sequence 0, 1, 1, 2, 3, 5, 8, 13, 21, 34, $\dots$ is defined recursively as
\begin{algorithm}[H]
	\caption{The Fibonacci sequence}\label{alg:fib}
	\begin{algorithmic}[1]
	\Function{FIB}{$n$}\Comment{This mathematical definition leads naturally to a recursive algorithm}
		\If{n $\leq$ 1}
			\State \Return n
		\Else
			\State \Return \Call{FIB}{n-1} + \Call{FIB}{n-2}
		\EndIf
	\EndFunction
	\end{algorithmic}
\end{algorithm}
	\begin{enumerate}[label=\arabic{enumi}\choicelabel]
	    \item Write the recurrence relation, $T(n)$, for the asymptotic runtime for procedure \Call{FIB}{$n$} shown above, and solve the recurrence relation to show that $T(n)=O\left( 2^{n-2} \right)$.
		\item Another recursive procedure which computes the $n$th Fibonacci number is below. \begin{algorithm}[H]
				\caption{}\label{alg:fib1}
				\begin{algorithmic}[1]
				\Function{F1}{$n$}
					\If{$n < 2$}
						\State \Return $n$
					\Else
						\State \Return \Call{F2}{2, $n$, 1, 1}
					\EndIf
				\EndFunction\\
				\Function{F2}{$i,n,x,y$}
					\If{$i \leq n$}
						\State \Call{F2}{$i+1$, $n$, $y$, $x+y$}
					\EndIf
					\State \Return x
				\EndFunction
				\end{algorithmic}
			\end{algorithm}
		\noindent Trace out the algorithm as it computes \Call{F1}{1}, \Call{F1}{2}, \Call{F1}{3}, \Call{F1}{4}, explain how the algorithm works, and then compare its asymptotic runtime to the time for procedure \Call{FIB}{$n$}.
	\end{enumerate}
	\item Use mathematical induction to show that when $n$ is an exact power $2$, the solution of the recurrence \[ T(n)=\left\{ \begin{array}{ll}
		2&$\mbox{ if }$ n=2\\
		2T(n/2)+n&$\mbox{ if }$ n=2^{k},\mbox{ for }k>1
	\end{array} \right. \] is $T(n)=n\lg n$
\end{enumerate}
%</Recitation-2>

\end{document}