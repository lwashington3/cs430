%! Author = Len Washington III
%! Date = 10/12/2023

% Preamble
\documentclass[12pt]{report}

\usepackage[8]{cs430recitation}
\usepackage{algpseudocode}

% Document
\begin{document}

%<*Recitation-8>
\subsection{After Lecture 15 \& 16} -- Answer any questions on HW4 (due today)\\
Practice Problems (all taken from previous exams)
\begin{enumerate}[label=\arabic*.]
	\item In dynamic programming, the technique of storing the previously calculated values is called \_\_\_\_\_\_\_\_\_\_\_\_
	\begin{enumerate}[label=\choicelabel]
	    \item Saving value property
		\item Storing value property
		\item Memoization
		\item Mapping
	\end{enumerate}
	\item What is the time complexity of the brute force algorithm used to find the longest common subsequence for sequence length $m$ and sequence length $n$ ($m < n$)?
	\begin{enumerate}[label=\choicelabel]
	    \item $O(mn)$
	    \item $O((mn)^{2})$
	    \item \answer{$O(n2^{m})$}
	    \item $O(2^{m}2^{n})$
	\end{enumerate}
	\item When dynamic programming is used, it takes less time compared to algorithmic methods that don't utilize overlapping subproblems.
	\begin{enumerate}[label=\choicelabel]
	    \item True.
		\item False.
	\end{enumerate}
	\item Using the dynamic programming solution, determine an LCS of $\{ 1, 0, 0, 1, 0, 1, 0, 1 \}$ and $\{ 0, 1, 0, 1, 1, 0, 1, 1, 0 \}$. Show all your work.
	\item Given a sequence of $n$ numbers $a_{1}$, $a_{2}$, $a_{3}$, $\dots$, $a_{n}$ (some of them might be negative) stored in an array, we want to find two indicies $i \leq j$ such that the sum of the numbers from $a_{i}$ to $a_{j}$ is maximum, among all possible $i$ $j$ pairs $1 \leq i \leq j \leq n$.
	\begin{enumerate}[label=\arabic{enumi}\alph*)]
	    \item Write pseudocode to sum each contiguous subsequence (from $a_{i}$ to $a_{j}$) and keep track of the maximum one. What is the runtime of your algorithm?
		\item Now find an $O(n)$ algorithm. Give pseudocode.
	\end{enumerate}
	\item Prove that a binary tree that is not full (every node has 0 or 2 children) cannot correspond to an optimal prefix code.
\end{enumerate}
%</Recitation-8>

\end{document}