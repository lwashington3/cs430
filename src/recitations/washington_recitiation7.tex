%! Author = Len Washington III
%! Date = 10/05/2023

% Preamble
\documentclass[12pt]{report}

\usepackage[7]{cs430recitation}
\usepackage{algpseudocode}

% Document
\begin{document}

%<*Recitation-7>
\subsection{After Lecture 13 \& 14}
Practice Problems (all taken from previous exams)
\begin{enumerate}[label=\arabic*.]
	\item If you want to create in order-statistic tree (which needs the size of each subtree rooted at each node), from an already created red-black tree, you can:
	\begin{enumerate}[label=\choicelabel]
	    \item perform a pre-order traversal of the order-statistic tree and sum the sizes of each subtree of a node and add one to get the size of each node (nodes with no children assigned size=1)
		\item perform an in-order traversal of the order-statistic tree and sum the sizes of each subtree of a node and add one to get the size of each node (nodes with no children assigned size=1)
		\item perform a post-order traversal of the order-statistic tree and sum the sizes of each subtree of a node and add one to get the size of each node (nodes with no children assigned size=1)
	\end{enumerate}
	\item How does an augmented data structure differ from a traditional data structure?
	\begin{enumerate}[label=\choicelabel]
	    \item Augmented data structures have an asymptotically higher memory overhead.
	    \item Augmented data structures worsen the asymptotic runtime of basic operations.
	    \item Augmented data structures offer additional operations or information.
	    \item Augmented data structures have a faster runtime complexity than the non-augmented data structure.
	\end{enumerate}
	\item If a problem can be broken into sub-problems which are reused several times, the problem has \_\_\_\_\_.
	\begin{enumerate}[label=\choicelabel]
	    \item Overlapping subproblems
		\item Optimal substructure
		\item Memoization
		\item Greedy
	\end{enumerate}
	\item What is the space complexity of the dynamic programming implementation of the matrix chain problem?
	\begin{enumerate}[label=\choicelabel]
	    \item $O(1)$
	    \item $O(n)$
	    \item $O(n^{2})$
	    \item $O(n^{3})$
	\end{enumerate}
	\item Given an element $x$ in an $n$-node order statistic tree and a natural number $i$, how can we determine the $i$th successor of $x$ in the linear order of the tree in $O(\lg n)$ time? So $x$ is a key in the tree and we want to find the $i$th key after $x$ in linear order. \orderstatistictree
	\item Suppose that the dimensions of the matrices $A$, $B$, $C$, and $D$ are $8\times5$, $5\times11$, $11\times6$, and $6\times9$ respectively, and that we want to parenthesize the product $ABCD$ in a way that minimizes the number of scalar multiplications. Find the $m$ and $s$ tables computed by MATRIX-CHAIN-ORDER to solve this problem and show the optimal parenthesization.
	\item Let $R(i,j)$ be the number of times that table entry $m[i,j]$ is referenced while computing other table entries in a call of MATRIX-CHAIN-ORDER. Show that the total number of references for the entire table is \[ \sum_{i=1}^{n}\sum_{j=i}^{n} R(i,j) = \frac{n^{3} - n}{3} \]
\end{enumerate}
%</Recitation-7>

\end{document}