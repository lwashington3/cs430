%! Author = Len Washington III
%! Date = 10/05/2023

% Preamble
\documentclass[7]{cs430recitation}

% Document
\begin{document}

%<*Recitation-7>
\subsection{After Lecture 13 \& 14}
Practice Problems (all taken from previous exams)
\begin{enumerate}[label=\arabic*.]
	\item If you want to create in order-statistic tree (which needs the size of each subtree rooted at each node), from an already created red-black tree, you can:
	\begin{enumerate}[label=\choicelabel]
	    \item perform a pre-order traversal of the order-statistic tree and sum the sizes of each subtree of a node and add one to get the size of each node (nodes with no children assigned size=1)
		\item perform an in-order traversal of the order-statistic tree and sum the sizes of each subtree of a node and add one to get the size of each node (nodes with no children assigned size=1)
		\item \oldanswer{perform a post-order traversal of the order-statistic tree and sum the sizes of each subtree of a node and add one to get the size of each node (nodes with no children assigned size=1)}
	\end{enumerate}
	\item How does an augmented data structure differ from a traditional data structure?
	\begin{enumerate}[label=\choicelabel]
	    \item Augmented data structures have an asymptotically higher memory overhead.
	    \item Augmented data structures worsen the asymptotic runtime of basic operations.
	    \item \oldanswer{Augmented data structures offer additional operations or information.}
	    \item Augmented data structures have a faster runtime complexity than the non-augmented data structure.
	\end{enumerate}
	\item If a problem can be broken into sub-problems which are reused several times, the problem has \_\_\_\_\_.
	\begin{enumerate}[label=\choicelabel]
	    \item \oldanswer{Overlapping subproblems}
		\item Optimal substructure
		\item Memoization\footnote{\textbf{Memoization} means that we should never try to compute the solution to the }
		\item Greedy
	\end{enumerate}
	\item What is the space complexity of the dynamic programming implementation of the matrix chain problem?
	\begin{enumerate}[label=\choicelabel]
	    \item $O(1)$
	    \item $O(n)$
	    \item \oldanswer{$O(n^{2})$}
	    \item $O(n^{3})$
	\end{enumerate}
	\item Given an element $x$ in an $n$-node order statistic tree and a natural number $i$, how can we determine the $i$th successor of $x$ in the linear order of the tree in $O(\lg n)$ time? So $x$ is a key in the tree and we want to find the $i$th key after $x$ in linear order. \orderstatistictree \oldanswer{First we determine the rank of $x$ by calling \Call{OS-RANK}{$T$, $x$} and name this number $r$. Then the $i$th successor of $x$ is actually an element in the tree with rank $r+i$. Hence we call \Call{OS-SELECT}{$T.root$, $r+i$}. Both these calls require $O(\lg n)$ time which in total is again $O(\lg n)$.}
	\item Suppose that the dimensions of the matrices $A$, $B$, $C$, and $D$ are $8\times5$, $5\times11$, $11\times6$, and $6\times9$ respectively, and that we want to parenthesize the product $ABCD$ in a way that minimizes the number of scalar multiplications. Find the $m$ and $s$ tables computed by MATRIX-CHAIN-ORDER to solve this problem and show the optimal parenthesization.\oldanswer{\begin{table}[H]
	    \centering
	    \begin{threeparttable}
			\caption{}
			\label{tab:}
			\begin{tabular}{lllll}
				$m$ & 	$A$	& 	$B$ 	& 	$C$		& 	$D$\\
				$A$ & 	0	&	$440$	&	$570$	& 	$960$\\
				$B$ & 		&	$0$		&	$330$	& 	$600$\\
				$B$ & 		&			&	$0$		& 	$594$\\
				$B$ & 		&			&			& 	$0$\\
			\end{tabular}
			\begin{tablenotes}
				\small
				\item
			\end{tablenotes}
		\end{threeparttable}
	\end{table}
	}
	\item Let $R(i,j)$ be the number of times that table entry $m[i,j]$ is referenced while computing other table entries in a call of MATRIX-CHAIN-ORDER. Show that the total number of references for the entire table is \[ \sum_{i=1}^{n}\sum_{j=i}^{n} R(i,j) = \frac{n^{3} - n}{3} \] \oldanswer{\begin{equation*}
	\begin{aligned}
		\sum_{i=1}^{n}\sum_{j=i}^{n} R(i,j) &= \sum_{l=2}^{n} \sum_{i=1}^{n-l+1} \sum_{k=i}^{i+l-2}2\\
											&= \sum_{l=2}^{n} \sum_{i=1}^{n-l+1} 2(l-1)\\
											&= \sum_{l=2}^{n} 2(l-1)(n-l+1)\\
											&= \sum_{l=1}^{n-1} 2l(n-1)\\
	\end{aligned}
	\end{equation*}}
\end{enumerate}
%</Recitation-7>

\end{document}