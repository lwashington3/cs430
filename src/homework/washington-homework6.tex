%! Author = Len Washington III
%! Date = 8/12/2023

% Preamble
\documentclass[6]{cs430homework}

% Document
\begin{document}
\maketitle
\begin{enumerate}[label=\arabic*.]
	\item (6 points) Assume you are creating an array data structure that has a fixed size of $n$. You want to backup this array after every so many insertion/update operations. Unfortunately, the backup operation is quite expensive, it takes $n$ time to do the backup, regardless of how many items are currently in the data structure. Insertions/updates without a backup just take 1 time unit.
	\begin{enumerate}[label=\arabic{enumi}\alph*)]
	    \item How frequently can you do a backup and still guarantee that the amortized cost of insertion/update is O(1)?
		\item Prove that you can do backups in O(1) amortized time.
	\end{enumerate}
	\item (7 points) Suppose we wish not only to increment a counter but also to reset it to zero (i.e., make all bits in it 0). Counting the time to examine or modify a bit as $\Theta$(1), show how to implement a counter as an array of bits so that any sequence of $n$ INCREMENT and RESET operations takes time $O(n)$ on an initially zero counter. You must use amortized analysis. (Hint: Keep a pointer to the high-order 1.)
	\item (7 points) \textbf{Rooted Fibonacci trees} $T_{n}$ are defined recursively in the following way. $T_{1}$ and $T_{2}$ are both the rooted tree consisting of a single vertex, and for $n=3,4,\dots$, the rooted tree $T_{n}$ is constructed from a root with $T_{n-1}$ as its left subtree and $T_{n-2}$ as its right subtree.
	\begin{enumerate}[label=\arabic{enumi}\alph*)]
	    \item Draw the first seven rooted Fibonacci trees.
		\item How many vertices, leaves, and internal vertices does the rooted Fibonacci tree $T_{n}$ have, where $n$ is a positive integer? What is its height?
	\end{enumerate}
	\item (7 points) Give an example of a series of \Call{Insert}{} and \Call{Extract-Min}{} operations on a Fibonacci Heap that will yield a heap of $n$ keys with height $n-1$.
	\item (6 points)
	\begin{table}[H]
	    \centering
	    \begin{threeparttable}
			\label{tab:}
			\begin{tabular}{|l|l|}
				\toprule
				\begin{minipage}[t]{0.5\textwidth}
				Show the data structure that results and the answers returned by the FIND-SET operations in the following program. Use the linked-list representation with the weighted-union heuristic.				\end{minipage}
				& \begin{minipage}[t]{0.5\textwidth}
					\begin{algorithmic}[1]
						\For{$i\gets1$ to $16$}
							\State \Call{Make-Set}{$x_{i}$}
						\EndFor
						\For{$i\gets1$ to $15$ by $2$}
							\State \Call{Union}{$x_{i}$,$x_{i+1}$}
						\EndFor
						\For{$i\gets1$ to $13$ by $4$}
							\State \Call{Union}{$x_{i}$,$x_{i+2}$}
						\EndFor
						\State \Call{Union}{$x_{1}$,$x_{5}$}
						\State \Call{Union}{$x_{11}$,$x_{13}$}
						\State \Call{Union}{$x_{1}$,$x_{10}$}
						\State \Call{Find-Set}{$x_{2}$}
						\State \Call{Find-Set}{$x_{9}$}
					\end{algorithmic}
				\end{minipage}
				\\\bottomrule
			\end{tabular}
			\begin{tablenotes}
				\small
				\item
			\end{tablenotes}
		\end{threeparttable}
	\end{table}

	\item (7 points) There is an image of ``$n$ by $m$'' pixels. Originally all are white, but then a few black pixels are drawn. You want to determine the size of each white connected component in the final image. Pixels are judged as connected if they share a side.
\end{enumerate}

\end{document}