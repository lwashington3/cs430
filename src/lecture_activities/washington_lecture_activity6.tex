%! Author = Len Washington III
%! Date = 8/12/2023

% Preamble
\documentclass[12pt]{report}

% Packages
\usepackage{cs430lecture}
\usepackage{algpseudocode}

% Document
\begin{document}

\listofalgorithms
%<*Lecture-Activity-6>
\setcounter{lecture}{6}
\section{Opening Questions}\label{sec:opening-questions}
\begin{enumerate}[label=\arabic*.]
    \item Mergesort is $\Theta(O(n\lg n))$ runtime in best case, worst case and average case. How much memory is needed for Mergesort on input size $n$? \answer{$O(n)$}
	\item Mergesort does all the work of sorting items in the Merge function, after recursively splitting the collection down to the base case. Briefly explain the difference with Quicksort.\answer{All the work for Mergesort is being done after the recurrsive calls, where as Quicksort does the work before the recurrsive calls.}
\end{enumerate}

\section{Quicksort}\label{sec:quicksort}
\noindent A recursive divide and conquer algorithm.
\begin{itemize}
	\item base case: a list of length one element is sorted.
	\item Divide ``Partition'' array into 2 sub-arrays with small \#'s in the beginning, large \#'s in second and known index dividing them\answer{ What defines small and large? The dividing index may not be the middle.}
	\item Conquer - recursively sort each sub-array
	\item Combine - Nothing to do
\end{itemize}
\begin{enumerate}[label=\arabic*.]
    \item Write pseudocode for Quicksort (similar to Mergesort).\begin{algorithm}[H]
    		\caption{Quicksort}\label{alg:quicksort}
    		\begin{algorithmic}[1]
    		\Function{QSort}{A, p, r}\Comment{The initial call is QSort(A, 1, n)}
				\If{p $<$ r}
					\State q $\gets$ Partition(A, p, r)\Comment{Moves small to left and large to right, q is the seperator}
					\State \Call{QSort}{A, p, q-1}\Comment{The value at A[q] is the pivot, q is the idnex}
					\State \Call{QSort}{A, q+1, r}
				\EndIf
    		\EndFunction
    		\end{algorithmic}
    	\end{algorithm}
\end{enumerate}

\noindent Partition idea (you should be able to do this in place): pick the last element in the current array as the ``pivot'', the number used to decide large or small. Then make a single pass of the array to move the ``small'' numbers before the ``large'' numbers and keep the ``large'' numbers after the ``small'' numbers. Then put the ``pivot'' between the two subarrays and return the location of the pivot.
\begin{enumerate}[label=\arabic*.,start=2]
    \item Write iterative pseudocode for Partition. How much memory is needed?\begin{algorithm}[H]
    		\caption{Partition: Worst case runtime $O(n)$}\label{alg:partition}
    		\begin{algorithmic}[1]
    		\Function{Partition}{A,p,r}
    			\State pivot $\gets$ A[r]
    			\State i $\gets p-1$
				\For{j=p to r-1}
					\If{A[j] $\leq$ pivot}
						\State Swap A[i+1] with A[j]
						\State increment i
					\EndIf
				\EndFor
			\State Swap A[i+1] with A[r]
			\State \Return i+1
    		\EndFunction
    		\end{algorithmic}
    	\end{algorithm} \hyperref[alg:quicksort]{Quicksort} Runtime: \begin{equation*}
    	\begin{aligned}
    		T(n) &= O(n) + T(\mbox{small items}) + T(\mbox{large items})\\
			T(1) &= O(1) + (O \leftrightarrow n-1) + (n-1 \leftrightarrow O)\\
			\mbox{If All Items} &< \mbox{Pivot}\\
				T(n) &= O(n) + T(n-1) + T(0)\\
					 &= O(1) + T(1-1) + O(1)\\
					 &= O(1) + T(0) + O(1)\\
					 &= O(1) \\
				T(n) &= O(n) + T(n-1)\\
					 &= O(n) + O(n-1) + T(n-2)\\
					 &= O(n) + O(n-1) + O(n-2) + T(n-3)\\
					 &= O(n) + O(n-1) + O(n-2) + O(n-3) + \dots + T(1)\\
					 &= O(n) + O(n-1) + O(n-2) + O(n-3) + \dots + O(1)\\
					 &= O\left( n^{2} \right)\\
			\mbox{If Pivot is Median}\\
				T(n) &= O(n) + 2T\left( \frac{n}{2} \right)\\
					 &(\mbox{Master method: } a=2, b=2, f(n)=O(n))\\
					 &= n^{\log_{b}(a)}\\
					 &= n^{\log_{2}(2)}\\
					 &= n^{1}\\
					 &= O(n\lg n)\\
			\mbox{What if pivot divides items} &\frac{1}{10} \mbox{ on oneside and } \frac{9}{10} \mbox{ on other}\\
				T(n) &= T\left( \frac{n}{10} \right) + T\left( \frac{n}{10} \right) + O(n)\\ % TODO: Do bubble method
					 & \dots\\
					 &= \log_{\frac{10}{9}}(n)\\
					 &= O(n\lg n)
    	\end{aligned}
    	\end{equation*} \answer{As long as the split is some sort of ratio, you get $O(n\lg n)$ as the runtime for quicksort. Only in the rare case of all the data being on one side of the partition that gives $O(n^{2})$ as the partition. To fix this, pick 3 items randomly, chose the middle of those 3, swap that middle with the last index and you'll be guaranteed to have $O(n\ln n)$.}\begin{algorithm}[H]
    		\caption{Partition Improved}\label{alg:partition-improved}
    		\begin{algorithmic}[1]
    		\Function{PartitionI}{A,p,r}
				\State $b$, $c$, $d$ are random indexes from A.
				\State Swap A[r] with \Call{Median}{A[b], A[c], A[d]}
				\State pivot $\gets$ A[r]
    			\State i $\gets p-1$
				\For{j=p to r-1}
					\If{A[j] $\leq$ pivot}
						\State Swap A[i+1] with A[j]
						\State increment i
					\EndIf
				\EndFor
			\State Swap A[i+1] with A[r]
			\State \Return i+1
    		\EndFunction
    		\end{algorithmic}
    	\end{algorithm}
%    		\caption{Partition}\label{alg:partition}
%    		\begin{algorithmic}[1]
%    		\Function{Partition}{A, i, r}
%				\For{i $\gets$ 1:n}
%					\State pivot $\gets$ A[r]
%					\If{A[j] $>$ pivot}
%						\State increment j
%					\EndIf
%					\If{A[j] $\leq$ pivot}
%						\State Swap A[i+1] with A[j]
%						\State increment i
%						\State increment j
%					\EndIf
%				\EndFor
%				\State Move pivot between small \& large
%    		\EndFunction
%    		\end{algorithmic}
%    	\end{algorithm}
	\item Demonstrate Partition on this array.
	\item What do you think the best possible outcome would be for a call to Partition, and why? What about worst possible outcome?
	\item Write (and solve) recurrence relations for Quicksort in the best case partition and worst case partition.
	\item What is there is a pretty bad, but not awful, partition at every call. Try always a 9 to 1 split from partition. Write and solve recurrence relation.
\end{enumerate}

\noindent With all the other sorts we could describe a particular input order that would yield worst case run time.
\begin{enumerate}[label=\arabic*.,start=7]
	\item How can we avoid a particular input order yielding worst case run time for quicksort? % TODO: We answered all of these questions above, organize the answers down.
\end{enumerate}

\noindent Visual Sorting Software By A. Alegoz, previous CS430 student (1.8Mb zipped, Win only)\url{http://www.cs.iit.edu/~cs430/IITSort.zip}
%</Lecture-Activity-6>
\end{document}