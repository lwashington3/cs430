%! Author = Len Washington III
%! Date = 9/12/2023

% Preamble
\documentclass[29]{cs430lecture}

% Packages

% Document
\begin{document}

%<*Lecture-Activity-29>
\maketitle

\section{All-Pairs Shortest Paths Problem}\label{sec:all-pairs-shortest-paths-problem}
\begin{itemize}
	\item Given a directed graph $G=(V,E)$, weight function $w: E\rightarrow R$, $|V|=n$,
	\item Goal: create an $n\times n$ matrix of shortest-path distances from every vertex to every other vertex $\delta(u, v)$,
	\item Could run \Call{Bellman-Ford} once from each vertex:
	\begin{itemize}
		\item $O(V^{2}E)$ which is $O(V^{4})$ if the graph is dense $(E\approxeq V^{2})$.
	\end{itemize}
	\item If no negative-weight edges, could run \hyperref[sec:dijkstra's-shortest-path-algorithm]{Dijkstra's algorithm} once from each vertex:
	\begin{itemize}
		\item $O(VE\lg V)$ with binary heap--$O(V^{3}\lg V)$ if dense.
	\end{itemize}
	\item We'll see how to do in $O(V^{3})$ in all cases with dynamic programming
	(we have already shown the shortest path problem has optimal substructure.)
\end{itemize}

The formal problem statement:
\begin{itemize}
	\item Assume that $G$ is given as an adjacency matrix of weights: $W=(w_{ij})$, with vertices numbered 1 to $n$.
	\[ w_{ij} = \left\{ \begin{array}{ll}
		0 & \mbox{if } i=j,\\
		\mbox{weight of } (i,j) & \mbox{if } i\neq j, (i,j)\in E,\\
		\infty & \mbox{if } i\neq j, (i,j)\notin E,\\
	\end{array} \right. \]
	\item Output is the shortest path matrix $D=(d_{ij})$, where $d_{ij}=\delta(i,j)$.
\end{itemize}

Dynamic Programming Steps
\begin{enumerate}
    \item Define structure of optimal solution, including what are the largest sub-problems.
	\item Recursively define optimal solution
	\item Compute solution using table bottom up
	\item Construct Optimal solution
\end{enumerate}

To help us develop the first dynamic programming approach, we can restate the All-Pairs Shortest Paths problems as follow.\\

Find the shortest path from every vertex to every other vertex considering at most paths of $|V|-1$ edges
(longest simple path for $|V|$ vertices).

\begin{enumerate}
    \item Define structure of optimal solution.
	\item Recursively define optimal solution
\end{enumerate}

\section{Slow All-Pairs Shortest Paths Algorithm}\label{sec:slow-all-pairs-shortest-paths-algorithm}
\begin{algorithm}[H]
	\caption{Slow All-Pairs Shortest Paths Algorithm}\label{alg:}
	\begin{algorithmic}[1]
	\State Compute a solution bottom-up:
	Compute $L^{(1)}=W,$ then $L^{(2)}$ from $L^{(1)}$, etc$\dots,L^{(n-1)}$
	\Function{Extend}{$L$, $W$, $n$}
		\State $L'\gets$ an $n\times n$ matrix
		\For{$i\gets1$ to $n$}
			\For{$j\gets1$ to $n$}
				\State $L'_{ij}\gets\infty$
			\EndFor
			\For{$k\gets1$ to $n$}
				\State $L'_{ij}\gets\min(L_{ij}',\ L_{ik} + W_{kj})$
			\EndFor
		\EndFor
		\State \Return $L'$
	\EndFunction\\
	\Function{Slow-APSP}{$W$, $n$}
		\State $L^{(1)}\gets W$
		\For{$m\gets 2$ to $n-1$}
			\State $L^{(m)}\gets\Call{Extend}{L^{(m-1)},W,n}$
		\EndFor
		\State \Return $L^{(n-1)}$
	\EndFunction
	\end{algorithmic}
\end{algorithm}

\begin{enumerate}[start=3]
    \item What is the runtime of \Call{Extend}{} and \Call{Slow-ASPS}{}?
\end{enumerate}

\section{Improving on SLOW-ASPS}\label{sec:improving-on-slow-asps}
Note the code to multiply two $n\times n$ matrices $(AB)$ together to get $C$, an $n\times n$ matrix.
\begin{algorithm}[H]
	\caption{Multiply Matricies}\label{alg:multiply-matricies}
	\begin{algorithmic}[1]
	\Function{Multiply-Matricies}{$A$, $B$}
		\For{$i\gets1$ to $n$}
			\For{$j\gets1$ to $n$}
				\State $C_{ij}\gets0$
				\For{$j\gets1$ to $n$}
					\State $C_{ij}\gets C_{ij} + A_{ik}B_{kj}$
				\EndFor
			\EndFor
		\EndFor
	\EndFunction
	\end{algorithmic}
\end{algorithm}

\begin{answer}
	\begin{table}[H]
		\centering
		\begin{threeparttable}
			\caption{The shortest path containing two edges.}
			\label{tab:slow-apsp-l2}
			\begin{tabular}{c|ccccc}
				$L^{(2)}$ & A & B & C & D & E\\
				\midrule
				A & 0 & $3_{A}$ & $8_{A}$ & $2_{E}$ & $-4_{A}$\\
				B & $3_{D}$ & 0 & $-4_{D}$ & 1 & 7\\
				C & $\infty$ & 4 & 0 & $5_{B}$ & $11_{B}$\\
				D & 2 & $-1_{C}$ & -5 & 0 & $-2_{A}$\\
				E & $\infty$ & $\infty$ & $\infty$ & 6 & 0
			\end{tabular}
			\begin{tablenotes}
				\small
				\item Augmented with $\infty$ when no edge exists
			\end{tablenotes}
		\end{threeparttable}
	\end{table}
\end{answer}

%\begin{answer}
%	\begin{minipage}[b]{0.33\textwidth}
%		\begin{table}[H]
%			\centering
%			\begin{threeparttable}
%				\caption{Adjacency matrix for the \hyperref[fig:floyd-warshall-example]{below graph}.}
%				\label{tab:slow-apsp-adjacency-matrix}
%				\begin{tabular}{c|ccccc}
%					W & A & B & C & D & E\\
%					\midrule
%					A & 0 & 3 & 8 & $\infty$ & -4\\
%					B & $\infty$ & 0 & $\infty$ & 1 & 7\\
%					C & $\infty$ & 4 & 0 & $\infty$ & $\infty$\\
%					D & 2 & $\infty$ & -5 & 0 & $\infty$\\
%					E & $\infty$ & $\infty$ & $\infty$ & 6 & 0
%				\end{tabular}
%				\begin{tablenotes}
%					\small
%					\item Augmented with $\infty$ when no edge exists
%				\end{tablenotes}
%			\end{threeparttable}
%		\end{table}
%	\end{minipage}
%	\begin{minipage}[b]{0.33\textwidth}
%		\begin{table}[H]
%			\centering
%			\begin{threeparttable}
%				\caption{The shortest path containing 1 edge (which is why it's the same as the adjacency matrix).}
%				\label{tab:slow-apsp-l1}
%				\begin{tabular}{c|ccccc}
%					$L^{(1)}$ & A & B & C & D & E\\
%					\midrule
%					A & 0 & $3_{A}$ & $8_{A}$ & $\infty$ & $-4_{A}$\\
%					B & $\infty$ & 0 & $\infty$ & 1 & 7\\
%					C & $\infty$ & 4 & 0 & $\infty$ & $\infty$\\
%					D & 2 & $\infty$ & -5 & 0 & $\infty$\\
%					E & $\infty$ & $\infty$ & $\infty$ & 6 & 0
%				\end{tabular}
%				\begin{tablenotes}
%					\small
%					\item Augmented with $\infty$ when no edge exists
%				\end{tablenotes}
%			\end{threeparttable}
%		\end{table}
%	\end{minipage}
%	\begin{minipage}[b]{0.33\textwidth}
%		\begin{table}[H]
%			\centering
%			\begin{threeparttable}
%				\caption{The shortest path containing two edges.}
%				\label{tab:slow-apsp-l2}
%				\begin{tabular}{c|ccccc}
%					$L^{(2)}$ & A & B & C & D & E\\
%					\midrule
%					A & 0 & $3_{A}$ & $8_{A}$ & $2_{E}$ & $-4_{A}$\\
%					B & $3_{D}$ & 0 & $-4_{D}$ & 1 & 7\\
%					C & $\infty$ & 4 & 0 & $5_{B}$ & $11_{B}$\\
%					D & 2 & $-1_{C}$ & -5 & 0 & $-2_{A}$\\
%					E & $\infty$ & $\infty$ & $\infty$ & 6 & 0
%				\end{tabular}
%				\begin{tablenotes}
%					\small
%					\item Augmented with $\infty$ when no edge exists
%				\end{tablenotes}
%			\end{threeparttable}
%		\end{table}
%	\end{minipage}
%\end{answer}

\begin{enumerate}[start=4]
    \item How does this matrix multiply code compare to the \Call{Extend}{} code?
	Why do we care?
	\begin{answer}$O(|V|^{4})$\end{answer}
\end{enumerate}

\section{Faster All-Pairs Shortest Paths Algorithm}\label{sec:faster-all-pairs-shortest-paths-algorithm}
\begin{algorithm}[H]
	\caption{Faster All-Pairs Shortest Paths Algorithm}\label{alg:faster-apsp}
	\begin{algorithmic}[1]
	\State Compute a solution bottom-up:
	Compute $L^{(1)}=W,$ then $L^{(2)}$ from $L^{(1)}$, then $L^{(4)}$ from $L^{(2)}$, etc$\dots,L^{(n-1)}$
	\Function{Faster-APSP}{$W$, $n$}
		\State $L^{(1)}\gets W$
		\State $m\gets1$
		\While{$m < n-1$}
			\State $L^{(2m)}\gets\Call{Extend}{L^{(m)},L^{(m)},n}$
			\State $m\gets2m$
		\EndWhile
		\State \Return $L^{(m)}$
	\EndFunction
	\end{algorithmic}
\end{algorithm}

\begin{enumerate}[start=5]
    \item What is the runtime of \Call{Faster-ASPS}{}?
	\begin{answer}$O(|V|^{3}\lg |V|)$\end{answer}
\end{enumerate}

\section{Floyd-Warshall Algorithm}\label{sec:floyd-warshall-algorithm}
To help us develop another dynamic programming approach,
we can state the All-Pairs Shortest Paths problem as follows:\\
Find the shortest path from every vertex to every other vertex considering at most all other vertices intermediate on the paths.

\begin{enumerate}[start=6]
    \item Define structure of optimal solution.
	\begin{answer}
		Assume optimal shortest path with possibly all other vertices along the path at most $|V|-2$.
		Remove a vertex from $k$ along the path, $k$ used or $k$ not user.
	\end{answer}
	\item Recursively define optimal solution and write pseudocode.
	\item What is the run time of \Call{Floyd-Warshall}{}?
	\item Demonstrate \Call{Floyd-Warshall}{}.
\end{enumerate}

\begin{figure}[H]
	\centering
	\begin{tikzpicture}[scale=1.5]
		\begin{scope}\globalnodeset
			\node (1) at (-3,-2) {A};
			\node (2) at (0,0) {B};
			\node (3) at (3,-2) {C};
			\node (4) at (2,-6) {D};
			\node (5) at (-2,-6) {E};
		\end{scope}
		\begin{scope}\globalpathset
			\path [->] (1) edge node {$3$} (2);
			\path [->] (1) edge[near end] node {$8$} (3);
			\path [->] (1) edge node {$-4$} (5);

			\path [->] (2) edge[near end] node {$1$} (4);
			\path [->] (2) edge[near end] node {$7$} (5);

			\path [->] (3) edge node {$4$} (2);

			\path [->] (4) edge[near end] node {$2$} (1);
			\path [->] (4) edge node {$-5$} (3);

			\path [->] (5) edge node {$6$} (4);
		\end{scope}
	\end{tikzpicture}
	\label{fig:floyd-warshall-example}
\end{figure}

%</Lecture-Activity-29>

\end{document}