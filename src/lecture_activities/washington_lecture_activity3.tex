%! Author = Len Washington III
%! Date = 8/12/2023

% Preamble
\documentclass[12pt]{report}

% Packages
\newcommand{\lectureactivity}{3}
\usepackage{cs430lecture}
\usepackage{algorithmicx}
\usepackage{algpseudocode}

% Document
\begin{document}

\section{Asymptotic Analysis (more details)}\label{sec:asymptotic-analysis-(more-details)}
\subsection{BIG-O Notation} -- Upper bound on growth of a runtime function
\begin{itemize}
	\item $f(n)\in O(g(n))$ reads as ``f(n) is big-O of g(n)''
\end{itemize}
If there exists $C$, $n_{0}$ such that $0 \leq f(n) \leq CG(n)$ when $n \geq n_{0}$ and $c>0$.	% FIXME: Change all the > and < to \leq and \geq

\begin{enumerate}[label=]
	\item
    \begin{enumerate}[label=\arabic{enumi}\alph*.]
        \item Use the definition of big-O to show $2n^{2}$ is big-O $n^{3}$ (find a C and $n_{0}$ that works in the above) \answer{\begin{equation*}
        \begin{aligned}
        	f(n)&<cg(n)\\
        	2n^{2}&\leq cn^{3}\mbox{ when }n>n_{0} (n>1) \mbox{ is } 2n^{2}<cn^{3}\\
			2(2^{2}) &\leq c(2^{3})	% TODO: Get the rest of this
        \end{aligned}
        \end{equation*}}
		\item Use the definition of big-O to show $T(n)=3n^{3}-4n^{2}+3\lg n-n=O(n^3)$ \answer{\begin{equation*}
		\begin{aligned}
			0 &< 3n^{3}-4n^{2}+3\lg n-n < cn^{3}(\mbox{ find } c>0 \mbox{ and } n_{0}>0 \mbox{ such that } n>n_{0})\\
			3n^{3}-4n^{2}+3\lg n-n ?&< 3n^{3}\\
			-4n^{2}+3\lg n-n ?&< 0\\
			\forall n &> 0, -4n^{2}\mbox{ is negative}
		\end{aligned}
		\end{equation*}Note: The $?<$ means is this true}
    \end{enumerate}
\end{enumerate}

\noindent
\subsection{Omega $\Omega$ Notation -- Lower Bound}
$g(n)=\Omega(g(n))\\
0 \leq cg(n) \leq f(n)\\
c?\mbox{~~~~~~~~~}n_{0} < n$

\begin{enumerate}[label=\arabic*., start=2]
    \item Use the definition of omega to show $n^{\frac{1}{2}}=\Omega(\log n)$ \answer{\begin{equation*}
    \begin{aligned}
		n > n_{0}, c > 0\\
    	c\lg(n) &\leq n^{\frac{1}{2}}\\
		n=4, c\log_{2}(4) ?\leq 4^{\frac{1}{2}}\\
		2c \leq 2\\
		n=16, c\log_{2}(16) ?\leq 16^{\frac{1}{2}}\\
		4c \leq 4\\
		2^{clog_{2}n} \leq 2^{\sqrt{n}}\\
		2^{c}2^{log_{2}n} \leq 2^{\sqrt{n}}\\
		n2^{c} \leq 2^{\sqrt{n}}\\
    \end{aligned}
    \end{equation*}}
\end{enumerate}

\subsection{Theta $\theta$ Notation -- Strict Bound}
$f(n)=\theta(g(n))\\
c_{1}g(n) \leq f(n) \leq c_{2}g(n)\\
c_{1}?<c_{2}?$

\begin{enumerate}[label=,start=3]
    \item \begin{enumerate}[label=\arabic{enumi}\alph*.]
        \item Use the definition of theta to show $3n^{3}-4n^{2}+37n=\theta(n^{3})$ \answer{\begin{equation*}
        \begin{aligned}
        	c_{1}n^{3} &\leq 3n^{3}-4n^{2}+37n &\leq c_{2}n^{3}\\
        	2n^{3} &?\leq 3n^{3}-4n^{2}+37n \\
        	0 &?\leq n^{3}-4n^{2}+37n \\
			\mbox{True for } n &> 0\\
			3n^{3}-4n^{2}+37n &\leq c_{2}n^{3}\\
			3n^{3}-4n^{2}+37n &\leq 3n^{3}\\
			-4n^{2}+37n &\leq 0\\
			n(-4n+37) &\leq 0\mbox{ } \forall n \geq 10\\
			c_{1}=2, n_{1}=0, c_{2}=3, n_{2}=10
        \end{aligned}
        \end{equation*}}
        \item Use the definition of theta to show $n^{2}+3n^{3}=\theta(n^{3})$ \answer{\begin{equation*}
        \begin{aligned}
        	c_{1}n^{3} &\leq n^{2}+3n^{3} &\leq c_{2}n^{3}\\
        	c_{1}n &\leq 1+3n \leq& c_{2}n\\
			c_{1} = 2 & & c_{2}=4 \mbox{ }\forall n>1
        \end{aligned}
        \end{equation*}}
    \end{enumerate}
\end{enumerate}

\section{Recursive Sorting -- Mergesort}
\begin{itemize}
	\item divide and conquer (and combine) approach, recursive algorithm
	\item key idea: you can merge sorted lists of total length $n$ in $\theta(n)$ linear time
	\item base case: a list of length one element is sorted
	\answer{\item For all sorting algorithms, the base case is $n=1$.}
\end{itemize}

\begin{enumerate}[label=\arabic*.]
    \item Demonstrate how you can merge two sorted sub-lists total $n$ items with $n$ compares/copies. How much memory do we need to do this? Write pseudocode to do this.
	\begin{table}[H]
	    \centering
	    \begin{threeparttable}
			\label{tab:merge-sort-numbers}
			\begin{tabular}{p{0.075\textwidth}p{0.075\textwidth}p{0.075\textwidth}p{0.25\textwidth}p{0.075\textwidth}p{0.075\textwidth}p{0.075\textwidth}p{0.075\textwidth}}	 2 & 3 & 7 & 8 & 1 & 4 & 5 & 6 \\
			\end{tabular}
		\end{threeparttable}
	\end{table}
	\begin{table}[H]
	    \centering
	    \begin{threeparttable}
			\label{tab:merge-sort}
			\begin{tabular}{|p{0.075\textwidth}|p{0.075\textwidth}|p{0.075\textwidth}|p{0.075\textwidth}|p{0.075\textwidth}|p{0.075\textwidth}|p{0.075\textwidth}|p{0.075\textwidth}|}
				\toprule
				 &  &  &  &  &  &  &  \\
				\bottomrule
			\end{tabular}
		\end{threeparttable}
	\end{table}

	\begin{algorithm}[H]
		\caption{Merge algorithm}\label{alg:merge}
		\begin{algorithmic}[1]
		\Function{Merge}{$A, i, j, k$}\Comment{$i$ is the index of the lower sorted sequence, $j$ is the index of the upper sorted, $k$ is the end}
			\State B $\gets$ array of size k
			\State l $\gets$ 0
			\State i\_end $\gets$ i-1
			\While{i $<$ i\_end \&\& j $\leq$ k}
				\If{$A[i] < A[j]$}
					\State B[l] $\gets$ A[i]
					\State i++
					\State l++
				\Else
					\State B[l] $\gets$ A[j]
					\State j++
					\State l++
				\EndIf
			\EndWhile\label{alg:merge-while}	% TODO: Finish Merge algorithm
		\EndFunction
		\end{algorithmic}
	\end{algorithm}


	\begin{algorithm}[H]
		\caption{Merge sort algorithm}\label{alg:merge-sort}
		\begin{algorithmic}[1]
		\Function{Mergesort}{$A, p, r$}\Comment{Initial call Mergesort($A, 1, n$)}
			\If{$p<r$}
				\State $q\gets\frac{p+r}{2}$\Comment{Integer division}
				\State MergeSort${(A, p, q)}$\Comment{Recursively sort 1st half}
				\State MergeSort${(A, q+1, r)}$\Comment{Recursively sort 2nd half}
				\State Merge${(A, p, q, r)}$\Comment{Merge 2 sorted sub-lists}
			\EndIf\label{alg:merge-sort-if}
		\EndFunction
		\end{algorithmic}
	\end{algorithm}
	\item Demonstrate Mergesort on this data:
	\begin{table}[H]
	    \centering
	    \begin{threeparttable}
			\label{tab:merge-sort-numbers-2}
			\begin{tabular}{p{0.01\textwidth}|p{0.12\textwidth}p{0.12\textwidth}p{0.12\textwidth}p{0.145\textwidth}p{0.12\textwidth}p{0.12\textwidth}p{0.12\textwidth}p{0.12\textwidth}}
					& 3 & 41 & 52 & 26 & 38 & 57 & 09 & 49 \\
				1. 	& 3(p=1, q=4) & 41 & 52 & 26 & 38 & 57 & 09 & 49(r=8) \\
				2. 	& 3(p=1, q=2) & 41 & 52 & 26(r=4) & & & & \\
				3. 	& 3(p=1, q=1) & 41 (r=2) \\
			\end{tabular}
		\end{threeparttable}
		\begin{tablenotes}
			\small
			\item I don't even know how I can write this part in \LaTeX
		\end{tablenotes}
	\end{table}
\end{enumerate}

\end{document}