%! Author = Len Washington III
%! Date = 10/21/2023

% Preamble
\documentclass[20]{cs430lecture}

% Packages

% Document
\begin{document}

%<*Lecture-Activity-20>
\definecolor{seagreen}{HTML}{008a87}
\maketitle
\openingquestions

\begin{enumerate}
	\item How do you think the allocated size growth of a dynamic array like Java's ArrayList is implemented?
	How much bigger does it grow when needed?
	What is the runtime for a sequence of $n$ insertions starting from a default size of 10 considering the worst individual insert?
	\begin{answer}The first few inserts run in $O(1)$ time in the continious memory that was initially assigned.
	When the array runs out of memory, it has to copy all of the current elements into a new, larger set of memory.
	The amount of time it would take to copy everything fron the initial array would be $O(\#\ of\ copies)$.
	If the worst case append is $O(n)$, then the runtime would be $nO(n) \rightarrow O(n^{2})$.\end{answer}
\end{enumerate}

\section[Amortized Analysis]{Amortized (to pay off gradually) Analysis}\label{sec:amortized-(to-pay-off-gradually)-analysis}
So far, we have analyzed best and wort case running times for an operation without considering its context.
With amortized analysis, we study a sequence of operations rather than individual operations.
An amortized analysis is any strategy for analyzing a sequence of operations to show that the average cost per operation is small, even though a single operation within the sequence might be expensive.

\section{Aggregate Method of Amortized Analysis}\label{sec:aggregate-method-of-amortized-analysis}
\begin{enumerate}
    \item Can we do a better analysis by amortizing the cost over all inserts?
	Starting with a table size one and doubling the size when necessary, make a table showing the first 10 inserts and determine a formula for \Call{cost}{$i$} for the cost of the $i$th insert.
	Then aggregate ``add up'' all the costs and divide by $n$ (aggregate analysis).
	\begin{answer}
		\begin{table}[H]
			\centering
			\begin{threeparttable}
				\caption{Aggregate Analysis for Appends}
				\label{tab:aggregate-analysis-for-appends}
				\begin{tabular}{|r|c|c|c|c|c|c|c|c|c|c|}
					\toprule
					\textbf{Append \#:} & 1 & 2 & 3 & 4 & 5 & 6 & 7 & 8 & 9 & 10\\
					\midrule
					\textbf{Memory Size} & 1 & 2 & 4 & 4 & 8 & 8 & 8 & 8 & 16 & 16\\
					\midrule
					\textbf{Total Cost (units)} & 1 & 2 & 3 & 1 & 5 & 1 & 1 & 1 & 9 & 1\\
					\midrule
					\textbf{Memory Cost (units)} &   & 1 & 2 &   &   &   &   &   & 8 &  \\
					\bottomrule
				\end{tabular}
				\begin{tablenotes}
					\small
					\item The memory cost is the cost in that append of creating a larger volume, it was created by removing the $O(1)$ cost of appending form the total cost row. Any iteration that didn't create more memory is 0, but is shown as blank.
				\end{tablenotes}
			\end{threeparttable}
		\end{table}
		\begin{equation}
		\begin{aligned}
			&\mbox{Append Cost per Operation} + \mbox{Memory Allocation Cost}\\
			&\sum_{i=1}^{n}1 + \sum_{j=0}^{\lg n} 2^{j}\\
			&n + \sim 2n = \theta(n)
		\end{aligned}\label{eq:amortized-cost}
		\end{equation}The total cost for $n$ appends amortized average cost is $\frac{\theta(n)}{n} = \theta(1)$.
	\end{answer}
\end{enumerate}

\section{Accounting Method of Amortized Analysis}\label{sec:accounting-method-of-amortized-analysis}
Figure out a specific amortized cost to be allocated to each operation to ensure you have enough ``balance'' to handle the bad operations.\\

Charge $i$th operation a fictitious amortized cost $\hat{c}_{i}$, where \$1 pays for 1 unit of work (i.e., time).
\begin{itemize}
	\item This fee is consumed to perform the operation.
	\item Any amount not immediately consumed is stored in the bank for use by subsequent operations.
	\item The bank balance must not go negative! We must ensure that for all $n$ \textcolor{seagreen}{\[ \sum_{i=1}^{n}c_{i} \leq \sum_{i=1}^{n}\hat{c}_{i} \]}
\end{itemize}
Thus, the amortized costs provide an upper bound on the total true costs.

\begin{enumerate}[start=2]
    \item For the previous ArrayList example, determine the amortized cost $\hat{c}_{i}$ necessary.
	\oldanswer{If we give \$3 per append and each assignment costs \$1. Series of $n$ append (some need growth).
	\begin{table}[H]
	    \centering
	    \begin{threeparttable}
			\caption{Accounting Analysis for Appends}
			\label{tab:accounting-analysis-for-appends}
			\begin{tabular}{|r|c|c|c|c|c|c|c|c|c|c|}
				\toprule
				\textbf{Append \#:} & 1 & 2 & 3 & 4 & 5 & 6 & 7 & 8 & 9 & 10\\
				\midrule
				\textbf{Insert (\$):} & +3 & +3 & +3 & +3 & +3 & +3 & +3 & +3 & +3 & +3\\
				\midrule
				\textbf{Assignment (Out \$)} & 1 & 2 & 3 & 1 & 5 & 1 & 1 & 1 & 9 & 1\\
				\midrule
				\textbf{Balance After:} & 2 & 3 & 3 & 5 & 3 & 5 & 7 & 9 & 3 & 5\\
				\bottomrule
			\end{tabular}
			\begin{tablenotes}
				\small
				\item We chose \$3 as the insert ``fee'' as the coefficient of $n$ we found earlier was \hyperref[eq:amortized-cost]{$\sim3$}.
			\end{tablenotes}
		\end{threeparttable}
	\end{table}}
	\item \begin{minipage}{0.7\textwidth}
	Consider, as a second example, a binary counter that is being implemented in hardware.
	Assume that the machine on which it is being run can flip a bit as its basic operation.
	We now want to analyze the cost of counting up from 0 to $n$ (using $k$ bits).\\

	What is the naive worst-case analysis for how many bits we need to flip?
	\oldanswer{Assuming you have some set of bits that starts with a 0 and every other bit is a 1 $01_{1}1_{2}1_{3}1_{4}\dots1_{k-1}$ (where $k-1$ is only the number of 1's), the next flip would flip all $k$ bits, so we would have $k$ operations $\dots$}
	\end{minipage}\begin{minipage}{0.3\textwidth}
		\begin{table}[H]
		    \centering
		    \begin{threeparttable}
				\label{tab:}
				\begin{tabular}{|c|c|}
					\toprule
					Decimal & Binary\\
					1 & 000001\\
					2 & 000010\\
					3 & 000011\\
					4 & 000100\\
					5 & 000101\\
					$\dots$ & $\dots$\\
					$n$ & 100110 \\
					\bottomrule
				\end{tabular}
			\end{threeparttable}
		\end{table}

	\end{minipage}
	\item Use the \hyperref[sec:aggregate-method-of-amortized-analysis]{aggregate method} to perform a more careful analysis for $n$ increments of a binary counter.
	\oldanswer{
	\begin{table}[H]
	    \centering
	    \begin{threeparttable}
			\caption{Amortized Cost of Flipping bits}
			\label{tab:amortized-cost-bit-flip}
			\begin{tabular}{|c|c|c|c|c|c|c|}
				\toprule
				1 & 10 & 11 & 100 & 101 & 110 & 111\\
				\midrule
				$0\rightarrow1$ & $1\rightarrow2$ & $2\rightarrow3$ & $3\rightarrow4$ & $4\rightarrow5$ & $5\rightarrow6$ & $6\rightarrow7$\\
				\midrule
				1 & 2 & 1 & 3 & 1 & 2 & 1\\
				\bottomrule
			\end{tabular}
		\end{threeparttable}
	\end{table}
	\begin{itemize}
		\item How often does the lowest bit flip? \textbf{Flips on every increment (n)}
		\item 2nd lowest bit? $\mathbf{\frac{n}{2}}$ times.
		\item 3rd lowest bit? $\mathbf{\frac{n}{4}}$ times.
		\item $\vdots$
		\item $k$th leftmost bit? $\frac{n}{2^{k-1}}$
	\end{itemize}
	\begin{equation*}
	\begin{aligned}
		&n + \frac{n}{2} + \frac{n}{4} + \dots + \frac{n}{2^{k-1}} \leq 2n\\
		&\mbox{Amortized Average Cost for $n$ increments } = \frac{2n}{n} = O(1)
	\end{aligned}
	\end{equation*}
	}
	\item Use the \hyperref[sec:accounting-method-of-amortized-analysis]{accounting method} to perform a more careful analysis for $n$ increments of a binary counter.
	\begin{answer}
		Price for each increment = \$2. Cost to flip a bit = \$1
		\begin{table}[H]
			\centering
			\begin{threeparttable}
				\caption{Accounting Cost of Flipping bits}
				\label{tab:accounting-cost-bit-flip}
				\begin{tabular}{|r|c|c|c|c|c|c|c|c|}
					\toprule
					& 1 & 10 & 11 & 100 & 101 & 110 & 111 & 1000\\
					\midrule
					\textbf{Counter} & $0\rightarrow1$ & $1\rightarrow2$ & $2\rightarrow3$ & $3\rightarrow4$ & $4\rightarrow5$ & $5\rightarrow6$ & $6\rightarrow7$ & $7\rightarrow8$\\
					\midrule
					\textbf{In (Price)} & 2 & 2 & 2 & 2 & 2 & 2 & 2 & 2\\
					\midrule
					\textbf{Out Cost}	& 1 & 2 & 1 & 3 & 1 & 2 & 1 & 4\\
					\midrule
					\textbf{Balance}	& 1 & 1 & 2 & 1 & 2 & 2 & 3 & 1\\
					\bottomrule
				\end{tabular}
				\begin{tablenotes}
					\small
					\item
				\end{tablenotes}
			\end{threeparttable}
		\end{table}
	\end{answer}
\end{enumerate}
%</Lecture-Activity-20>

\end{document}