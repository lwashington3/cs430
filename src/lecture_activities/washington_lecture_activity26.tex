%! Author = Len Washington III
%! Date = 11/12/2023

% Preamble
\documentclass[26]{cs430lecture}

% Document
\begin{document}

%<*Lecture-Activity-26>
\maketitle
\openingquestions
\begin{enumerate}
    \item What is the difference between a tree and a graph?
    \answer{All trees are graphs, but not all graphs are trees.
    In a tree, there is only 1 path between any 2 vertices.
    Trees are also acyclic since no item can have multiple parents.}
    \item Give a recursive definition for a tree.
    \answer{Base case: single node with no children.
    A tree is $\dots$ a node pointing to other trees with no cycles.}
    \item In a weighted undirected graph, what is the difference between a minimum spanning tree and a shortest path in a graph?
    \item Since the shortest paths contain the shortest sub-paths (\hyperref[dfn:optimal-substructure]{optimal substructure}),
    name an algorithmic approach that we might try to find a shortest path in a graph.
\end{enumerate}

\section{Minimum Spanning Trees (MST)}\label{sec:minimum-spanning-trees}
\begin{enumerate}
    \item Give a definition of a Minimum Spanning Tree, and find an MST of the below graph.
    \answer{Set of edges that connects all vertices with no cycles.
        $|V|-1$ edges; not necessarily unique, you could have multiple MSTs for a graph.}
    \begin{figure}[H]
        \centering
        \begin{tikzpicture}
            \begin{scope}[every node/.style={circle,thick,draw}]
                \node (A) at (0,0) {A};
                \node (D) at (1,-2.5) {D};
                \node (F) at (4,-2.75) {F} ;
                \node (B) at (8,0.5) {B};
                \node (C) at (2.5,1) {C};
                \node (E) at (4.25,-0.5) {E};
            \end{scope}
            \begin{scope}[>={Stealth[black]},
                every node/.style={fill=white,circle},
                every edge/.style={draw=black,very thick}]
                \path (A) edge[bend left=60] node {$5$} (B);
                \path (A) edge node {$3$} (C);
                \path (A) edge node {$9$} (D);

                \path (B) edge node {$2$} (C);
                \path (B) edge node {$1$} (E);
                \path (B) edge node {$6$} (F);

                \path (C) edge node {$4$} (E);

                \path (D) edge node {$7$} (E);

                \path (E) edge node {$4$} (F);
            \end{scope}
        \end{tikzpicture}
        \label{fig:26.1}
    \end{figure}
    \item Prove a Minimum Spanning Tree has \hyperref[dfn:optimal-substructure]{optimal substructure}.
    \answer{Pick any subsets of adjacent vertices in an optimal MST.
    Those edges that connect that subset of vertices in the MST must also be MST}
    \item What are some possible greedy approaches to find a Minimum Spanning Tree?
    Prove correct or show counterexample.
    \answer{\begin{itemize}
                \item Grow min edges first; no cycles.
                \item Prune/remove max edge, but stay connected.
                \item Creating edges from visited nodes to unvisited nodes.
                \item \textbf{Prim}\label{dfn:prim}: Min edge from visited vertex set to unvisited vertex set.
                \item \textbf{Kruskal}\label{dfn:kruskal}:
                Pick min edge that its vertices are not already in the \hyperref[sec:strongly-connected-components]{connected component}.
    \end{itemize}}
    \item Demonstrate your MST algorithm on the following graph and write pseudocode.
    \begin{figure}[H]
        \centering
        \begin{tikzpicture}
            \begin{scope}[every node/.style={circle,thick,draw}]
                \node (a) at (0,0) {a};

                \node (b) at (3,1.5) {b};
                \node (h) at (3,-1.5) {h};

                \node (i) at (5,0) {i};

                \node (c) at (7,1.5) {c};
                \node (g) at (7,-1.5) {g};

                \node (d) at (11,1.5) {d};
                \node (f) at (11,-1.5) {f};

                \node (e) at (14,0) {e};
            \end{scope}

            \begin{scope}[>={Stealth[black]},
                every node/.style={fill=white,circle},
                every edge/.style={draw=black,very thick}]
                \path (a) edge node {$4$} (b);
                \path (a) edge node {$8$} (h);

                \path (b) edge node {$11$} (h);
                \path (b) edge node {$8$} (c);

                \path (h) edge node {$7$} (i);
                \path (h) edge node {$1$} (g);

                \path (i) edge node {$2$} (c);
                \path (i) edge node {$6$} (g);

                \path (c) edge node {$7$} (d);
                \path (c) edge node {$4$} (f);

                \path (g) edge node {$2$} (f);

                \path (d) edge node {$14$} (f);
                \path (d) edge node {$9$} (e);

                \path (f) edge node {$10$} (e);
            \end{scope}
        \end{tikzpicture}
        \label{fig:26.2}
    \end{figure}
    \answer{\begin{table}[H]
                \centering
                \begin{threeparttable}
                    \label{tab:mst-example}
                    \begin{tabular}{c|c}
                        \textbf{Node} & \textbf{Visited?}\\
                        \midrule
                        $a$ & \\
                        $b$ & \\
                        $c$ & Visited\\
                        $d$ & \\
                        $e$ & \\
                        $f$ & \\
                        $g$ & \\
                        $h$ & \\
                        $i$ & \\
                    \end{tabular}
                    \begin{tablenotes}
                        \small
                        \item Started at $c$.
                    \end{tablenotes}
                \end{threeparttable}
    \end{table}
    }
\end{enumerate}

\answer{\begin{minipage}{0.5\textwidth}
    \begin{algorithm}[H]
        \caption{Prim's Algorithm (MST)}\label{alg:mst-prim}
        \begin{algorithmic}[1]
        \Function{MST-Prim}{$G$, $w$, $r$}
            \ForAll{$u\in G.V$}
                \State $u.key\gets\infty$
                \State $u.\pi \gets $ NIL
            \EndFor
            \State $r.key\gets0$
            \State $Q\gets G.B$
            \While{$Q\neq\emptyset$}
                \State
            \EndWhile
        \EndFunction
        \end{algorithmic}
    \end{algorithm}
\end{minipage}\begin{minipage}{0.5\textwidth}
\begin{algorithm}[H]
	\caption{Kruskal's Algorithm (MST)}\label{alg:mst-kruskal}
	\begin{algorithmic}[1]
	\Function{MST-Kruskal}{}
        \State
	\EndFunction
	\end{algorithmic}
\end{algorithm}\end{minipage}}

Demonstration of Prim (Deleted): \url{http://en.wikipedia.org/wiki/File:Prim-algorithm-animation-2.gif}

Demonstration of Kruskal: \url{https://www.cs.usfca.edu/~galles/visualization/Kruskal.html}
%</Lecture-Activity-26>

\end{document}