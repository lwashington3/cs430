%! Author = Len Washington III
%! Date = 11/12/2023

% Preamble
\documentclass[26]{cs430lecture}

% Packages

% Document
\begin{document}

%<*Lecture-Activity-26>
\maketitle
\openingquestions
\begin{enumerate}
    \item What is the difference between a tree and a graph?
    \item Give a recursive definition for a tree.
    \item In a weighted undirected graph, what is the difference between a minimum spanning tree and a shortest path in a graph?
    \item Since the shortest paths contain the shortest sub-paths (\hyperref[dfn:optimal-substructure]{optimal substructure}),
    name an algorithmic approach that we might try to find a shortest path in a graph.
\end{enumerate}

\section{Minimum Spanning Trees}\label{sec:minimum-spanning-trees}
\begin{enumerate}
    \item Give a definition of a Minimum Spanning Tree, and find an MST of the below graph.
    \begin{figure}[H]
        \centering
        \begin{tikzpicture}
            \begin{scope}[every node/.style={circle,thick,draw}]
                \node (A) at (0,0) {A};
                \node (D) at (1,-2.5) {D};
                \node (F) at (4,-2.75) {F} ;
                \node (B) at (8,0.5) {B};
                \node (C) at (2.5,1) {C};
                \node (E) at (4.25,-0.5) {E};
            \end{scope}

            \begin{scope}[>={Stealth[black]},
                every node/.style={fill=white,circle},
                every edge/.style={draw=black,very thick}]
                \path (A) edge[bend left=60] node {$5$} (B);
                \path (A) edge node {$3$} (C);
                \path (A) edge node {$9$} (D);

                \path (B) edge node {$2$} (C);
                \path (B) edge node {$1$} (E);
                \path (B) edge node {$6$} (F);

                \path (C) edge node {$4$} (E);

                \path (D) edge node {$7$} (E);

                \path (E) edge node {$4$} (F);
            \end{scope}
        \end{tikzpicture}
        \label{fig:26.1}
    \end{figure}
    \item Prove a Minimum Spanning Tree has \hyperref[dfn:optimal-substructure]{optimal substructure}.
    \item What are some possible greedy approaches to find a Minimum Spanning Tree?
    Prove correct or show counterexample.
    \item Demonstrate your MST algorithm on the following graph and write pseudocode.
    \begin{figure}[H]
        \centering
        \begin{tikzpicture}
            \begin{scope}[every node/.style={circle,thick,draw}]
                \node (a) at (0,0) {a};

                \node (b) at (3,1.5) {b};
                \node (h) at (3,-1.5) {h};

                \node (i) at (5,0) {i};

                \node (c) at (7,1.5) {c};
                \node (g) at (7,-1.5) {g};

                \node (d) at (11,1.5) {d};
                \node (f) at (11,-1.5) {f};

                \node (e) at (14,0) {e};
            \end{scope}

            \begin{scope}[>={Stealth[black]},
                every node/.style={fill=white,circle},
                every edge/.style={draw=black,very thick}]
                \path (a) edge node {$4$} (b);
                \path (a) edge node {$8$} (h);

                \path (b) edge node {$11$} (h);
                \path (b) edge node {$8$} (c);

                \path (h) edge node {$7$} (i);
                \path (h) edge node {$1$} (g);

                \path (i) edge node {$2$} (c);
                \path (i) edge node {$6$} (g);

                \path (c) edge node {$7$} (d);
                \path (c) edge node {$4$} (f);

                \path (g) edge node {$2$} (f);

                \path (d) edge node {$14$} (f);
                \path (d) edge node {$9$} (e);

                \path (f) edge node {$10$} (e);
            \end{scope}
        \end{tikzpicture}
        \label{fig:26.2}
    \end{figure}
\end{enumerate}

Demonstration of Prim (Deleted): \url{http://en.wikipedia.org/wiki/File:Prim-algorithm-animation-2.gif}

Demonstration of Kruskal: \url{https://www.cs.usfca.edu/~galles/visualization/Kruskal.html}
%</Lecture-Activity-26>

\end{document}