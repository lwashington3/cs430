%! Author = Len Washington III
%! Date = 10/1/2023

% Preamble
\documentclass[12pt]{report}

% Packages
\usepackage[14]{cs430lecture}

% Document
\begin{document}

%<*Lecture-Activity-14>
\section{Opening Questions}\label{sec:opening-questions-14}
\begin{enumerate}[label=\arabic*.]
    \item Briefly explain what two properties a problem must have so that a dynamic programming algorithm will work.
	\item Previously we have learned that \hyperref[divide-and-conquer]{divide-and-conquer algorithms} partition a problem into independent sub-problems, solve each sub-problem recursively, and then combine their solutions to solve the original problem. Briefly, how are dynamic programming algorithms similar and how are they different from divide-and-conquer algorithms?
	\item Why does it matter how we parenthesize a chain of matrix multiplications? We get the right answer any way we associate the matrices for multiplication. i.e.\ If $A$, $B$ and $C$ are matrices of correct dimensions for multiplication, then $(A\times B)C = A(B\times C)$.
\end{enumerate}

\section{Dynamic Programming}\label{sec:dynamic-programming}
Dynamic Programming Steps
\begin{enumerate}[label=\arabic*.,leftmargin=0.75in]
	\item Define structure of optimal solution, including what are the largest sub-problems.
	\item Recursively define optimal solution
	\item Compute solution using table bottom up
	\item Construct Optimal Solution
\end{enumerate}

\subsection{Optimal Matrix Chain Multiplication (optimal parenthesization)}\label{subsec:optimal-matrix-chain-multiplication-(optimal-parenthesization)}
\begin{enumerate}[label=\arabic*.]
    \item How many ways are there to parenthesize (two at a time multiplication) 4 matrices $A\times B\times C\times D$?
	\item Step 1: Generically define the structure of the optimal solution to the Matrix Chain Multiplication problem. The optimal way to multiply $n$ matrices $A_{1}$ through $A_{n}$ is:
	\item\label{prb:3} Step 2: Recursively define the optimal solution. Assume $P(1,n)$ is the optimal cost answer. Make sure you include the base case.
	\item Use proof by contradiction to show that Matrix Chain Multiplication problem has optimal substructure, i.e. the optimal answer to the problem must contain optimal answers to sub-problems.
	\item Step 3: Compute solution using a table bottom up for the Matrix Chain Multiplication problem. Use you answer to question \hyperref[prb:3]{3} above. Note the overlapping sub-problems as you go.
	\item Step 4: Construct Optimal solution
	\begin{table}[H]
	    \centering
	    \begin{threeparttable}
			\label{tab:}
			\begin{tabular}{llll}
				$A$ & $B$ & $C$ & $D$\\
				$2\times 4$ & $4\times 6$ & $6\times 3$ & $3\times 7$\\
			\end{tabular}
		\end{threeparttable}
	\end{table}

\end{enumerate}
%</Lecture-Activity-14>

\end{document}