%! Author = Len Washington III
%! Date = 9/12/2023

% Preamble
\documentclass[28]{cs430lecture}

% Packages

% Document
\begin{document}

%<*Lecture-Activity-28>
\maketitle
\openingquestions
\begin{itemize}
	\item We saw the \hyperref[sec:shortest-path-algorithm-bellman-ford]{Bellman-Ford} algorithm found the
	shortest path from a source to all other vertices by ``brute force''
	every edge in the graph in a fixed order $|V|-1$ times.
	Why did it need to do this $|V|-1$ times?
	And with this in mind, could we improve on the Bellman-Ford for certain graphs?
	\begin{newanswer}
		At least relax edges leaving from source first.
		Possible that the shortest path $u\leadsto v$ goes through every other vertex $|V|-1$ edges might be relaxed in opposite order.
	\end{newanswer}
\end{itemize}

\section{DAG Shortest Path Algorithm}\label{sec:dag-shortest-path-algorithm}
By relaxing the edges of a weighed DAG (directed acyclic graph) $G=(V,E)$ in topological sort order of its vertices, we can compute shortest paths from a single source.
Shortest paths are always defined in a DAG, since even if there are negative-weight edges, no negative weight cycles can exist.
\begin{algorithm}[H]
	\caption{DAG Shortest Path\begin{newanswer}$O(V^{2})$\end{newanswer}}\label{alg:dag-shortest-path}
	\begin{algorithmic}[1]
	\Function{DAG-Shortest-Path}{$G$, $w$, $s$}
		\State topologically sort the vertices of $G$
		\State \Call{\hyperref[alg:init-single-source]{Init-Single-Source}}{$G$, $s$}
		\ForAll{vertex $u$, taken in topologically sorted order}
			\State \ForAll{vertex $v\in Adj[u]$}
				\State \Call{\hyperref[alg:relax]{Relax}}{$u$, $v$, $w$}
			\EndFor
		\EndFor
	\EndFunction
	\end{algorithmic}
\end{algorithm}

\begin{enumerate}
    \item Here is the topological sort on a DAG\@.
	Find the shortest path from $s$ to every other vertex.
	\item What is the runtime for DAG Shortest Path?
	\begin{newanswer}$O(V+E)\Rightarrow O(V+V^{2})\Rightarrow O(V^{2})$\end{newanswer}
	\item Discuss why DAG Shortest Path is correct.
	\item If we restrict the graph to having no negative edges, given a source $s$, what is the shortest path from $s$ to one of its adjacent vertices?
\end{enumerate}

\section{Dijkstra's Shortest Path Algorithm}\label{sec:dijkstra's-shortest-path-algorithm}
\begin{itemize}
	\item No negative-weight \emph{edges}.
	\item Essentially a weighted version of breadth-first search.
	\begin{itemize}
		\item Instead of a FIFO queue, uses a priority queue.
		\item Keys are shortest-path weight estimates ($d[v]$).
	\end{itemize}
\end{itemize}

Dijkstra's Algorithms\\
\url{https://www.youtube.com/watch?v=wtdtkJgcYUM}
\url{https://www.cs.usfca.edu/~galles/visualization/Dijkstra.html}
%</Lecture-Activity-28>

\end{document}